\HeaderA{gpslope}{A Gaussian process prediction reference on slopes}{gpslope}
%
\begin{Description}\relax
A Gaussian process prediction reference on slopes
\end{Description}
%
\begin{Usage}
\begin{verbatim}
gpslope(data, x_, y_, trainingpect = 0.1, seed = 123, iter = 100,
  chains = 4, conf = 0.5, condition1 = NULL, condition2 = NULL,
  dn = 64, rho_alpha = 4, rho_beta = 4, alpha_mean = 0, alpha_sd = 1,
  sigma_mean = 0, sigma_sd = 1)
\end{verbatim}
\end{Usage}
%
\begin{Arguments}
\begin{ldescription}
\item[\code{data}] A list of input values

\item[\code{x\_}] name of x variable, should match with the data frame variable name

\item[\code{y\_}] name of y variable, should match with the data frame variable name

\item[\code{trainingpect}] the percent of training instances, the default value is 10\% of the whole sample, avoid high number of training instances when you want the stan program to be done faster

\item[\code{seed}] a random generator seed to designate such that your results will be the same across different runs

\item[\code{iter}] Number of iterations you would like stan to run

\item[\code{chains}] Number of chains you would like stan to utilize

\item[\code{conf}] Confidence level resulting in the interval displayed for the first derivative of the gaussian process prediction

\item[\code{condition1}] The first condition you want the first derivative to satisfy, such as ">0", the default value is NULL

\item[\code{condition2}] The second condition you want the first derivative to satisfy, such as "<2", the default value is NULL

\item[\code{dn}] The number you want to sample for density calculation for the first derivative's distribution for each x*, the default value is 64

\item[\code{rho\_alpha}] Prior distribution of rho is set to inverse gamma distribution with rho \textasciitilde{} inv\_gamma(rho\_alpha, rho\_beta), default value is inv\_gamma(4, 4)

\item[\code{rho\_beta}] Prior distribution of rho is set to inverse gamma distribution with rho \textasciitilde{} inv\_gamma(rho\_alpha, rho\_beta), default value is inv\_gamma(4, 4)

\item[\code{alpha\_mean}] Prior distribution of alpha is set to normal distribution with alpha \textasciitilde{} normal(alpha\_mean, alpha\_sd), default value is normal(0, 1)

\item[\code{alpha\_sd}] Prior distribution of alpha is set to normal distribution with alpha \textasciitilde{} normal(alpha\_mean, alpha\_sd), default value is normal(0, 1)

\item[\code{sigma\_mean}] Prior distribution of sigma is set to normal distribution with sigma \textasciitilde{} normal(sigma\_mean, sigma\_sd), default value is normal(0, 1)

\item[\code{sigma\_sd}] Prior distribution of sigma is set to normal distribution with sigma \textasciitilde{} normal(sigma\_mean, sigma\_sd), default value is normal(0, 1)
\end{ldescription}
\end{Arguments}
%
\begin{Value}
A list consiting of the Stan fit object, three ggplot plot objects and the user-specified argument values
\end{Value}
%
\begin{Examples}
\begin{ExampleCode}
resultlst <- gpviz(data = hsbReport_cleaned, x_ = 'deposit_asset_ratio', y_ = 'loss_asset_ratio', seed = 123456, iter = 2000, conf = 0.95, condition1 = '>0', condition2 = NULL)
\end{ExampleCode}
\end{Examples}
